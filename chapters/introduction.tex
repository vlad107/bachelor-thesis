\startprefacepage

%Актуальность
Система Автоматического Доказательства Теорем (далее САДТ) -- популярная область математической логики, дающая обширные практические результаты. С помощью САДТ можно, в частности, помогать в доказательстве теорем из различных областей математики(например интерактивное программное средство доказательства теорем \emph{Coq}), тестировать интегральные схемы и верифицировать программный код (например \cite{Detlefs:2005:STP:1066100.1066102}, \cite{zap-automated-theorem-proving-for-software-analysis}) тестировать криптографические протоколы (например \cite{DBLP:journals/corr/MoranW17}).


Существует и ежегодно дополняется архив задач \emph{TPTP} \cite{Sutcliffe2009}. На нём вы можете найти примеры, затрагивающие огромное количество областей, в удобном для САДТ виде.


%Цель
Целью данной работы является обобщение уже существующего исчисления \emph{Conflict Resolution} \cite{DBLP:journals/corr/SlaneyP16} на модель Акторов. Рассмотрение различных подходов к оптимизации нового исчисления, а именно: динамическое распределение мощностей и добавление дополнительных ограничений на правила вывода, связанных с коммуникацией акторов.


%Новизна
Несмотря на большое количество плюсов акторной модели, ни один из существующих на сегодняшний день подходов для параллелизации систем автоматического доказательства теорем не использует её. Стоит, впрочем, упомянуть, что в обзорной работе \cite{Bonacina2018} был упомянут подход, использующий диффундирующие вычисления. 
%% TODO: ПОЯСНИТЬ чем диффундирующие вычисления близки к Акторам.

%Практическое значение
В акторной модели общение между вычислительными процессами(акторами) происходит при помощи асинхронного обмена сообщениями, которые посылаются от актора к актору. Отсутствие понятия \emph{общей памяти} в акторной модели позволяет нам не только распараллелить вычислительные процессы, но и предоставить возможность распределённых вычислений (например, при помощи фреймворка \emph{akka-remote}).


% Краткий обзор струткруы работы
%% глава 1
В главе \ref{sec:chap1} будет формально введена логика первого порядка, описана решаемая задача. Также будет введена базовая версия решения задачи, основанная на исчислении \emph{Conflict Resolution}. 


%% глава 2
В главе \ref{sec:chap2} будет рассказано обобщение исчисления \emph{Conflict Resolution} на акторную модель, доказана её полнота относительно опровержения. Будет предложено введение дополнительных ограничений на правила вывода с соответствующими доказательствами полноты относительно опровержения. Также будет рассказано об исчислении с динамическим изменением специализации акторов.

%% глава 3
В главе \ref{sec:chap3} вкратце будут описаны детали реализации, оценка производительности на наборе задач \emph{TPTP}, предоставлены графики для сравнения с предыдущими версиями.