\chapter{Введение в область}
\startrelatedwork
\label{sec:chap1}

\section{Исчисление предикатов первого порядка}

Далее мы опишем нотацию, предложенную Alexander Leitsch \cite{Leitsch:1997:RC:260906}.

\subsection{Определение исчисления}

Далее будем считать, что нам даны:
\begin{enumerate}
	\item $V$    --- счётное множество названий переменных
    \item $CS$   --- множество константных символов
    \item $FS_i$ --- множество $i$-нарных функциональных символов, где $i \in \mathbb{N}$. \\
    За $FS$ обозначим $\bigcup\limits_{i=1}^{\infty} FS_{i}$
    \item $PS_i$ --- множество $i$-нарных предикатных символов, где $i \in \mathbb{N}$. \\
    За $PS$ обозначим $\bigcup\limits_{i=1}^{\infty} PS_{i}$
    \item Логические связки --- символы $\vee, \wedge, \rightarrow, \neg$
    \item Кванторы --- символы $\forall, \exists$
\end{enumerate}



\begin{definition}
  \emph{Множество термов $T$.} Минимальное по включению множество, определяемое индуктивно:
  \begin{enumerate}
  	\item $V \subseteq T$
    \item $CS \subseteq T$
    \item Если $t_1, t_2, \ldots, t_k \in T; f \in FS_k$, тогда $f(t_1, t_2, \ldots, t_k) \in T$
  \end{enumerate}
  будем называть \emph{множеством термов}.
\end{definition}

\begin{definition}
  \emph{Атомарная формула.} Пусть $t_1, t_2, \ldots, t_k \in T; P \in PS_k$, тогда формула $P(t_1, t_2, \ldots, t_k)$ называется атомарной.
\end{definition}

Множество атомарных формул обозначим за $AT$. 

\begin{definition}
  \emph{Литерал.} Пусть $A$ -- атомарная формула, тогда $A, \neg A$ -- литералы.
\end{definition}

Множество литералов обозначим за $LIT$

Примеры литералов:
\begin{example}
Рассмотрим следующий литерал
$\neg ODD(mul(2,x))$.  
\begin{enumerate}
	\item $ODD$ --- предикат
    \item $mul$ --- функциональный символ
    \item $2$ --- константный символ
    \item $x$ --- переменная
\end{enumerate}
Если мы назначим естественную оценку этому литералу, то он будет отвечать на вопрос "правда ли, что для любого $x$, $2*x$ -- чётно". 
\end{example}


\begin{definition}
  \emph{Множество формул исчесления предикатов $PL$.} минимальное по включению множество, определяемое индуктивно:
  \begin{enumerate}
  	\item $LIT \subseteq PL$
    \item Пусть $A, B \in PL$, тогда $A \to B, A \vee B, A \wedge B \in PL$ 
    \item Пусть $A \in PL, x \in V, (\forall x) \notin A, (\exists x) \notin A$, тогда $(\forall x)A, (\exists x)A \in PL$
  \end{enumerate}
\end{definition}

Иначе говоря, формулы исчисления предикатов -- это литералы, связанные логическими операциями и кванторами.

\subsection{Интерпретация}

\begin{definition}
  \emph{Свободное вхождение.} Определим индуктивно: 
  \begin{enumerate}
  	\item Пусть формула $A \in LIT$. Тогда $x$ входит свободно, если $x$ встречается в $A$.
    \item Пусть формула $A = B \odot C$, где $\odot \in \{ \to, \vee, \wedge \}$. Тогда $x$ входит свободно в $A$, если $x$ входит свободно в $B$, либо $C$.
    \item Пусть формула $A = (Qy)B$, где $Q \in \{ \exists, \forall \}$. Тогда $x$ входит свободно в $A$, если $x$ входит свободно в $B$ и $y \neq x$
  \end{enumerate}
\end{definition}

\begin{definition}
  \emph{Замкнутое вхождение.} $x$ входит замкнуто в формулу $A$, если у $A$ существует подформула $B$ вида $(Qx)B'$, где $Q \in \{ \exists, \forall \}$ и $x$ входит свободно в $B'$
\end{definition}

\begin{definition}
  \emph{Открытая и замкнутые формулы.}
    \begin{enumerate}
      \item Если у формулы нет свободных вхождений переменных, то она называется \textit{замкнутой}.
      \item Если у формулы нет замкнутых вхождений переменных, то она называется \textit{открытой}.
    \end{enumerate}
\end{definition}

\begin{definition}
  \emph{Интерпретация формулы.} Интерпретацией формулы $F$ из $PL$ назовём тройку $\Gamma = (D, \Phi, I)$, для которой верно следующее:
    \begin{enumerate}
		\item $D$ --- непустое множество, которое называется доменом $\Gamma$
        \item $\Phi$ --- отображение из $CS(F) \cup FS(F) \cup PS(F)$, заданное так:
          \begin{enumerate}
              \item $\Phi(c) \in D$, если $c \in CS(F)$
              \item Если $f \in FS_n(F)$, то $\Phi(f) : D^n \rightarrow D$
              \item Если $P \in PS_n(F)$, то $\Phi(P) : D^n \rightarrow \{ 0, 1 \}$
          \end{enumerate}
        \item $I$ -- фукнция из $V$ в $D$, которая задаёт \textit{назначение переменных}
    \end{enumerate}

\end{definition}


\begin{definition}
  \emph{Эквивалентность интерпретаций.} Две интерпретации $\Gamma$ и $\Delta$ называются эквивалентными по модулю $x_1, x_2, \ldots, x_k$, если существуют такие $D, \Phi, I_1, I_2$, что $\Gamma = {(D, \Phi, I_1)}$, $\Delta = {(D, \Phi, I_2)}$, а также верно $v \notin \{x_1, x_2, \ldots, x_k\} \Rightarrow I_1(v) = I_2(v)$. Далее будем писать $\Gamma \sim_x \Delta$.
\end{definition}

\begin{definition}
  \emph{Оценка исчисления предикатов.} Введём оценку $\nu_{\Gamma}$ как функцию из формулы в $\{0, 1\}$, которая будет определяться индукцией по структуре:
  \begin{enumerate}
  	\item Если формула $A$ -- атомарная, т.е. имеет вид $P(t_1, t_2, \ldots, t_k)$, то $\nu_{\Gamma}(A) = \Phi(P)(\nu_{\Gamma}(t_1), \nu_{\Gamma}(t_2), \ldots, \nu_{\Gamma}(t_k))$
  	\item \begin{enumerate}
    		\item $\nu_\Gamma(A \vee B) = \nu_\Gamma(A) \vee \nu_\Gamma(B)$
    		\item $\nu_\Gamma(A \wedge B) = \nu_\Gamma(A) \wedge \nu_\Gamma(B)$
    		\item $\nu_\Gamma(A \rightarrow B) = \nu_\Gamma(A) \rightarrow \nu_\Gamma(B)$
    		\item $\nu_\Gamma(\neg A) = \neg(\nu_\Gamma(A))$
    	  \end{enumerate}
    \item $\nu_\Gamma((\forall x)A) = 1$, если для любой $\Delta$ верно $\Gamma \sim_x \Delta \iff \nu_\Delta(A) = 1$ 

 \item $\nu_\Gamma((\exists x)A) = 1$, если существует такая $\Delta$, что $\Gamma \sim_x \Delta$ и $\nu_\Delta(A) = 1$
  \end{enumerate}
\end{definition}

Будем говорить, что 
\begin{enumerate}
	\item Интерпретация $\Gamma$ \textit{верифицирует} формулу $A$, если $\nu_{\Gamma}(A) = 1$.
	\item Интерпретация $\Gamma$ \textit{опровергает} формулу $A$, если $\nu_{\Gamma}(A) = 0$.
\end{enumerate}

\subsection{Удовлетворимость и корректность}

\begin{definition}
  \emph{Модель.} Пусть дана формула $A$, содержащая свободные вхождения переменных $x_1, x_2, \ldots, x_k$, тогда интерпретация $\Gamma$ является \textit{моделью} для $A$, если для любой интерпретации $\Delta \sim_x \Gamma$ верно, что $\nu_{\Delta}(A) = 1$. Если $A$ -- замкнутая формула, то $\Gamma$ является моделью тогда и только тогда, когда $\Gamma$ верифицирует $A$.
\end{definition}

\begin{definition}
  \emph{Удовлетворимость и корректность.} 
  \begin{enumerate}
  	\item Формула $F$ называется \textit{удовлетворимой}, если для неё существует модель.
  	\item Формула $F$ называется \textit{корректной}, если любая интерпретация $\Gamma$ для этой формулы является моделью.
  	\item Формулы $F$ и $G$ \textit{логически эквивалентны}, если у них одинаковое множество моделей. Обозначение $F \sim G$
  	\item Будем писать $F \sim_{sat} G$, если $F \text{ -- удовлетворима} \iff G \text{ -- удовлетворима}$
  \end{enumerate}
\end{definition}

\begin{definition}
  \emph{Клоз.} Пусть $L_1, L_2, \ldots, L_k$ -- литералы, тогда формулу вида $L_1 \vee L_2 \vee \dots \vee L_k$ будем называть \emph{клозом}
\end{definition}

\begin{definition}
  \emph{Конъюктивная нормальная форма.} Пусть $C_1, C_2, \ldots, C_m$ --- \emph{клозы}, тогда формулу вида $C_1 \wedge C_2 \wedge \dots \wedge C_m$ находится в Конъюктивно Нормальной Форме(\emph{КНФ}).
\end{definition}

Заметим, что для любой формулы из \emph{Исчисления предикатов первого порядка}, можно найти эквивалентную ей формулу в \emph{КНФ}. Это делается при помощи \textit{Клозификации} \cite{clausification}. Стоит, впрочем, упомянуть, что размер эквивалентной формулы имеет экспоненциальный размер относительно оригинальной формулы исчисления.

Заметим также, что формула $A$ корректна тогда и только тогда, когда $\neg A$ не удовлетворима. А из наблюдений выше вопрос об удовлетворимости произвольной формулы эквивалентен вопросу об удовлетворимости множества клозов. 

В отличие от \textit{Исчисления Высказываний}, в \textit{Исчислении Предикатов} мы не можем проверить все интерпретации, так как, вообще говоря, их бесконечно много.\\

\section{Унификация}
Для формального введения унификации нам понадобятся дополнительные определения.

\begin{definition}
  \emph{Подстановка.} Отображение $\lambda$ из множества переменных во множество термов называется \emph{подстановкой}, если $\lambda(v) \neq v$ только для конечного количества переменных.
\begin{itemize}  
  \item \emph{Областью определения} подстановки называется множество переменных, которые отображаются не сами в себя, и обозначается за $dom(\lambda) = \{v | v \in V, \lambda(v) \neq v\}$ 
  \item \emph{Областью значений} называется множество термов, в которые отображается область определения, и обозначается за $rg(\lambda) = \{\lambda(v) | v \in dom(\lambda)\}$
\end{itemize}
\end{definition}

Иначе говоря, подстановка заменяет некоторые вхождения переменных на некоторые подтермы.

Принято использовать постфиксную нотацию для применения \emph{подстановки}, которая определяется следующим индуктивным образом: 
\begin{itemize}
	\item $x\lambda = \lambda(x)$, для $x \in V$
    \item $a\lambda = a$, для $a \in CS$
    \item $f(t_1, \ldots, t_n)\lambda = f(t_1\lambda, \ldots, t_n\lambda$, для $f \in FS; t_1,\ldots,t_n \in T$
    \item $P(t_1, \ldots, t_n)\lambda = P(t_1\lambda, \ldots, t_n\lambda)$, где $P(t_1, \ldots, t_n)$ -- атом.
    \item $(A \circ B)\lambda = A\lambda \circ B\lambda$, где $\circ \in \{\vee, \wedge, \rightarrow\}$
    \item $(\neg A)\lambda = \neg(A\lambda)$
    \item $((Qx)A)\lambda = (Qx)(A\lambda)$, если известно, что $x \notin dom(\lambda), Q \in \{\forall, \exists\}$
\end{itemize}

\begin{definition}
  \emph{Выражение.} Выражение --- это либо терм, либо атом, либо литерал.
\end{definition}

Пусть $W = \{E_1, E_2, \ldots, E_n\}$ --- множество выражений, тогда под $W\sigma$ мы будем понимать применение применение подстановки к каждому из выражений, т.е. $W\sigma = \{E_1\sigma, E_2\sigma, \ldots, E_n\sigma\}$.

Поймём как связаны выражения и подстановки:
\begin{definition}
\emph{Частное выражение.} Выражение $E_2$ является частным случаем выражения $E_1$, если существует такая подстановка $\sigma$, что $E_1\sigma = E_2$.
\end{definition}

Построим отношение частичного порядка на выражениях и подстановках:
\begin{definition}
\emph{Обобщённость.}
  \begin{enumerate}
	\item Будем говорить, что выражение $E_1$ не менее общее, чем выражение $E_2$, если существует такая подстановка $\sigma$, что $E_1\sigma = E_2$. Формально будем записывать так: $E_1 \leq_s E_2$.
    \item Для двух подстановок $\sigma$ и $\tau$ будем говорить, что $\sigma \leq_s \tau$, если существует такая подстановка $\vartheta$, что $\tau = \sigma\vartheta$. Иначе говоря, $\tau$ является композицией подстановок $\sigma$ и $\vartheta$.
  \end{enumerate}
\end{definition}

Теперь мы готовы определить унификацию.

\begin{definition}
\emph{Унификация.} Пусть $W$ --- непустое множество выражений. Подстановка $\sigma$ называется унификатором, если $|W\sigma| = 1$, то есть все выражения после применения подстановки становятся одинаковыми.
\end{definition}

\begin{definition}
\emph{Наиболее общий унификатор.} Унификатор $\sigma$ для множества выражений $W$ называется наиболее общим, если для любого другого унификатора $\tau$ верно, что $\sigma \leq_s \tau$.
\end{definition}

Рассмотрим пример для унификации:
\begin{example}
Для следующего набора атомов:
\begin{gather*}
\{ P(x, f(y)), P(x, f(x)), P(x',y'), P(x', f(y)) \}
\end{gather*}
унификатором является следующая подстановка
\begin{gather*}
\{ x \leftarrow t, y \leftarrow t, x' \leftarrow t, y' \leftarrow f(t) \}
\end{gather*}
Действительно, после применения этой подстановки все выражения сойдутся в $P(t, f(t))$. Заметим также, что данный унификатор является наиболее общим.
\end{example}

Поиск наиболее общих унификаторов является известной задачей. Наиболее популярный подход описан в статье Мартелли и Монтанари \cite{Martelli:1982:EUA:357162.357169}. Этот же подход используется в реализации, которая будет рассказана в главе \ref{sec:chap3}.

\section{Резолюции}

На сегодняшний день, большинство систем для автоматического доказательства теорем берёт за основание \emph{Резолюционное исчисление}. Наши рассуждения не являются исключением. Давайте вкратце введём базовое исчисление резолюций.

\subsection{Резолюционное исчисление}
\begin{figure}
\begin{calculus}
\centering
\textbf{Resolution:}
$$
\infer[\mathbf{r}(\sigma)]
	  {(\Gamma \vee \Delta)\sigma}
      {\Gamma \vee \ell & \overline{\ell'} \vee \Delta}
$$

где $\sigma$ --- унификатор для $\ell$ и $\ell'$.

\bigskip

\textbf{Factoring:}
$$
\infer[\mathbf{f}(\sigma)]
      {(\ell \vee \ell'_1 \vee \ldots \ell'_m)\sigma}
      {\ell_1 \vee \ldots \vee \ell_n \vee \ell'_1 \vee \ldots \ell'_m}
$$

где $\sigma$ --- унификатор для $\ell_1, \ldots, \ell_n$, при этом $\ell = \ell_k \sigma$ для любого $1 \leq k \leq n$.
\end{calculus}
\caption{Исчисление \emph{Resolution}}
\label{fig:R}
\end{figure}

Резолюционное исчисление описано на рисунке \ref{fig:R}. В нём всего два правила: резолюция и факторизация. Резолюция позволяет нам объединять два клоза в один. Факторизация позволяет \textquote{схлопывать} литералы внутри одного клоза.

Чаще всего вывод клоза представляется в виде графа, где в вершину входят рёбра из тех вершин, которые участвовали в данном правиле вывода. 

\usetikzlibrary{arrows.meta}
\begin{figure}
\begin{tikzpicture}
\begin{scope}[every node/.style={circle,thick,draw,minimum size=14mm}]
    \node (A) at (0,0)  {$P$};
    \node (B) at (3,2)  {$Q$};
    \node (C) at (3,-2) {$\neg Q$};
    \node (D) at (5,0) {$\bot$};
\end{scope}

\begin{scope}[>={Stealth[black]},
              every node/.style={fill=white,circle},
              every edge/.style={draw=red,very thick}]
    \path [->] (A) edge node {...} (B);
    \path [->] (A) edge node {...} (C);
    \draw [->] (B) edge (D);
    \path [->] (C) edge (D);
\end{scope}
\end{tikzpicture}
\caption{Пример графа доказательства}
\label{fig:graph-example}
\end{figure}

\begin{example}
Рассмотрим пример доказательства в виде графа на рисунке \ref{fig:graph-example}. Здесь из клоза $P$ выводятся два клоза $Q$ и $\neg Q$, после чего к ним применяется правило \emph{Resolution} из исчисления на рисунке \ref{fig:R}.
\end{example}

\subsection{Conflict Resolution}
\label{sec:CR}
Исчисление \emph{Conflict Resolution} является обобщением классических идей метода из Резолюционного исчисления \emph{CDCL} на теореию первого порядка. Обзор других подходов к решению данной задачи можно найти в работе \cite{DBLP:journals/corr/SlaneyP16}. 

На рисунке \ref{fig:CR} описаны правила вывода для исчисления \emph{Сonflict Resolution} \cite{DBLP:journals/corr/SlaneyP16}. Здесь за $\ell$ обозначаются литералы. Особенностью этого исчисления является возможность делать предположения, которые на рисунке берутся в квадратные скобки. Для удобства в формальных рассуждениях, предположения скрыты в правило вывода \emph{Conflict-Driven Clause Learning}.

\begin{figure}
\begin{calculus}
\centering
\textbf{Unit-Propagating Resolution:}
$$
\infer[\upr{}{\sigma}]
	  {\ell~\sigma}
      {\ell_1 & \ldots & \ell_n & \overline{\ell'_1} \vee \ldots \vee \overline{\ell'_n} \vee \ell}
$$

где $\sigma$ --- унификатор $\ell_k$ и $\ell'_k$ для всех $k \in \{1, \ldots, n \}$.

\bigskip

\textbf{Conflict:}
$$
\infer[\confl{}{\sigma}]
      {\bot}
      {\ell & \overline{\ell'}}
$$

где $\sigma$ --- унификатор $\ell$ и $\ell'$.

\bigskip

\textbf{Conflict-Driven Clause Learning:}
$$
\infer[\cdcl{}{i}]
      {(\overline{\ell_1} \sigma^1_1 \vee \ldots \vee \overline{\ell_1} \sigma^1_{m_1}) \vee \ldots \vee (\overline{\ell_n} \sigma^n_1 \vee \ldots \vee \overline{\ell_n} \sigma^n_{m_n})
      }
	  {\infer*{\bot}{\infer*[(\sigma_1^1,\ldots,\sigma_{m_1}^1)]{}{[\ell_1]^{\dls{i}{1}}} &  & \infer*[(\sigma_1^n,\ldots,\sigma_{m_n}^n)]{}{[\ell_n]^{\dls{i}{n}}}}
      }
$$

где $\sigma^k_j$ (для $1 \leq k \leq n$ и $1 \leq j \leq m_k$) -- это \\
композиция всех подстановок, использованных на $j$-ом пути \footnote{Так как
  доказательство -- это ориентированный ациклический граф, который возможно не
  является деревом, то может существовать несколько путей, соединяющих $\ell_k$ и
  $\bot$ в доказательстве.} от $\ell_k$ до $\bot$.

\end{calculus}
\caption{Исчисление \emph{Conflict Resolution}}
\label{fig:CR}
\end{figure}


\section{Акторы}

  Модель акторов, впервые была упомянута в \cite{Hewitt:1973:UMA:1624775.1624804}. По своей сути акторная модель --- это множество независимых вычислительных элементов, которые обмениваются информацией с помощью сообщений. Можно считать, что акторы -- вершины графа, а ориентированное рёбро от актора $a$ к актору $b$ говорит о том, что актор $a$ может посылать сообщения актору $b$. Таким образом акторная модель -- множество независимых вычислительных элементов с соответствующей топологией. 
  
Каждый актор может исполнять следующие операции:

\begin{enumerate}
  	\item $receive(msg)$ --- принимает сообщение $msg$ от другого актора и какую-то последовательность операций, которая зависит от $msg$.
    \item $send(actorId, msg)$ --- в процессе выполнения функции $fun$ из предыдущего пункта, актор может послать сообщение $msg$ другому актору с идентификатором $actorID$.
\end{enumerate}

Формальное введение акторов -- нетривиальная задача, требующая высокого уровня формализма. Однако заметим, что в исчислении, которое будет введено в главе \ref{sec:chap2} нам важны только несколько фактов:
\begin{enumerate}
	\item Каждый актор имеет свою независимую память -- это, в частности, позволяет назначать акторам некоторые функции.
    \item Актор умеет отправлять сообщения другим акторам, и принимать сообщения.
    \item Каждый актор имеет собственные инструкции выполнения программ. Эти инструкции зависят от получаемого сообщения.
    \item Можно изменять функции, задающие поведение акторов. \emph{(потребуется для динамической специализации)}.
\end{enumerate}

\section{Выводы к первой главе}
\begin{enumerate}
	\item Формально введена задача, решаемая в данной работе.
    \item Описан базовый подход к решению данной задачи --- резолюционное исчисление.
    \item Описан обобщённый подход исчисление \emph{Conflict Resolution}.
    \item Введена модель акторов, на которую исчсисление \emph{Conflict Resolution} будет обобщено в главе \ref{sec:chap2}.
\end{enumerate}

