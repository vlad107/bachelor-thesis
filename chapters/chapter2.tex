\chapter{Постановка задачи}
\startrelatedwork
\label{sec:chap2}


\section{Мотивация}
Систему для автоматического доказательства теорем можно рассматривать как совокупность небольших подсистем (акторов). В этой системе каждый актор:
\begin{enumerate}
	\item Получает некоторые формулы от других акторов
    \item Выделяет из них подформулы, на которых он специализируется (например только часть с предикатами из заданного множества)
    \item Делает выводы из формул по своей специализации
    \item сообщает эту информацию другим акторам (следующему по кругу, либо всем, либо по какому-нибудь заданному алгоритму)
\end{enumerate} \par
Интуитивная мотивация такого подхода происходит из социиологии, где люди не решают сложные задачи поодиночке. Например, если у меня болит горло, то я сделаю некоторые допущения о своём состоянии и обращусь к доктору. Таким образом, у каждой сущности есть небольшое подмножество задач, соответсвующих специализации этой сущности. А решение задачи происходит в процессе коммуникации сущностей.

\section{Формальное построение модели}

\subsection{Описание акторов}

Так как у каждого актора есть внутреннее состояние, то мы можем определять функции и предикаты, зависящие от номера актора. Давайте введём функцию, котороая будет распределять литералы по акторам, то есть отвечать на вопрос "Работает ли актор $A$ с литералом $\ell$".

\begin{definition}
  \emph{Специализация.} Пусть $A = \{A_1, A_2, \ldots, A_N\}$ --- множество из $N$ акторов. 
  Тогда семейство предикатов $\{\Psi_{i}\}_{i \in \{1,\ldots,N\}}$, заданных на множестве литералов $LIT$, 
  называется \emph{специализацией} актора, если для любого литералла $\ell$ существует актор $i$ такой, что $\Psi_i(\ell)$ \par
  
  Будем говорить, что актор $i$ \emph{специализируется} на литерале $\ell$, если верно $\insp{i}{\ell}$.
\end{definition}

На данном этапе будем считать, что \emph{специализация} задана заранее и не меняется в процессе доказательства. Однако, в разделе \ref{sec:onlineChanging} мы рассмотрим вариант с двумя дополнительными правилами вывода, который подразумевает изменение \emph{специализации} в процессе доказательства.

\begin{definition}
	\emph{Остаточный клоз.} Пусть $C = \ell_1 \vee \ldots \vee \ell_n$ -- клоз. Тогда $rem_a(C) = \ell_{i_1} \vee \ldots \vee \ell_{i_k}$ -- остаточный клоз относительно актора $a$, если $\Psi_a(\ell_i) \iff i \in \{i_1, \ldots, i_k\}$
\end{definition}

%% TODO: изменение этой функции в процессе доказательства, т.к. у акторов есть функция become, которая изменяет поведение актора.

\subsection{Описание Expertised Conflict Resolution исчисления}

На рисунке \ref{fig:ECR} показаны правила для обобщённого на распределённый случай исчисления \emph{Conflict Resolution}, которое было описано в секции \ref{sec:CR}. \par

На правила вывода накладываются следующие дополнительные условия:
\begin{enumerate}
	\item \emph{Unit-Propagation Resolution.}
    \begin{itemize}
    	\item ${\ell^i_1, \ldots, \ell^i_n, \ell'^i_1, \ldots, \ell'^i_n, \ell}$ -- литералы из специализации актора $i$, т.е. для любого $k$ верно, что $\insp{i}{\ell^i_k}$, $\insp{i}{\ell'^i_k}$, $\insp{i}{\ell^i}$
        \item $\sigma$ -- унификатор для $\ell^i_k$ и $\ell'^i_k$ при всех $k \in \{1, \ldots, n \}$ 
        \item $\jmath^i, \jmath_1^i, \ldots, \jmath_n^i$ -- остаточные клозы соответствующих высказываний из предпосылки.
    \end{itemize}
        
	\item \emph{Cofnlict.}
    \begin{itemize}
        \item $\ell^i, \ell'^i$ -- литералы из актора $i$, т.е. верно $\insp{i}{\ell^i_k}$ и $\insp{i}{\ell'^i_k}$
    	\item $\sigma$ -- унификатор $\ell$ и $\ell'$
        \item $\jmath_1^i, \jmath_2^i$ -- остаточные клозы соответствующих высказываний из предпосылки.
    \end{itemize}
        
    \item \emph{Conflict-Driven Clause Learning.}
    \begin{itemize}
    	\item ${\ell^i_1, \ldots, \ell^i_n, \ell'^i_1, \ldots, \ell'^i_n, \ell}$ -- литералы из актора $i$, т.е. для любого $k$ верно, что $\insp{i}{\ell^i_k}$, $\insp{i}{\ell'^i_k}$, $\insp{i}{\ell^i}$
        
		\item $\overline{\insp{i}{\jmath^i_k}}$ для всех $k \in \{1, \ldots, m\}$ 

		\item $\sigma^k_j$ (для $1 \leq k \leq n$ и $1 \leq j \leq m_k$) -- это
композиция всех подстановок, использованных на $j$-ом пути \footnote{Так как
  доказательство -- это ориентированный ациклический граф, который возможно не
  является деревом, то может существовать несколько путей, соединяющих $\ell^i_k$ и
  $\jmath$ в доказательстве.} от $\ell^i_k$ до $\jmath = \jmath^i_1 \vee \ldots \vee \jmath^i_m$
  
  		\item  $\sigma'^k_j$ (для $1 \leq k \leq m$ и $1 \leq j \leq m'_k$) -- это
композиция всех подстановок, использованных на $j$-ом пути от $\jmath^i_k$ до $\jmath$
    \end{itemize}
    
    \item \emph{Konwledge sharing.}
    На данном этапе не будем описывать дополнительных условий для этого правила вывода, однако будут рассмотрены идеи ограничений, которые помогут нам ускорить реализацию данного алгоритма.
\end{enumerate}

В отличие от классического исчисления \emph{Conflict Resolution}, здесь каждому клозу приписывается ещё и номер актора, в котором на данный момент этот клоз был выведен. Правило \emph{Knowledge sharing} позволяет изменять этот номер. В предпосылках у старых правил \emph{Unit-Propagation}, \emph{Conflict} и \emph{Conflict-Driven Clause Learning} теперь вместо чистых литералов фигурируют целые клозы, которые, однако, имеют специальный вид: $\ell \vee \jmath$, где $\ell$ -- литерал, а $\jmath$ -- остаточный клоз соответствующего актора.



\begin{figure}
  \begin{prooftree}
    \AxiomC{$(A \vee B)^1$}
    \AxiomC{$\neg A^1$}
    \RightLabel{$\confl{1}{}$}
    \BinaryInfC{$B^1$}
    \RightLabel{$\kshare{1}{2}$}
    \UnaryInfC{$B^2$}
    \AxiomC{$\neg B^2$}
    \RightLabel{$\confl{2}{}$}
    \BinaryInfC{$\bot$}
  \end{prooftree}
  \caption{Пример взаимодействия акторов}
  \label{fig:ecr-example-1}
\end{figure}

\begin{figure}
  \begin{prooftree}
    \AxiomC{$[P(x)]^1$}
    
	\AxiomC{$\neg P(f(y)) \vee Q$}
    \RightLabel{$\upr{}{\{x \setminus f(y)\}}$}
    \BinaryInfC{$Q$}
    
    \AxiomC{$[P(x)]^1$}
    \AxiomC{$\neg P(f(z)) \vee \neg Q$}
    \RightLabel{$\upr{}{\{x \setminus f(z)\}}$}
    \BinaryInfC{$\neg Q$}
    
    \RightLabel{$\confl{}{}$}
    \BinaryInfC{$\bot$}
    
    \RightLabel{$\cdcl{}{}$}
    \UnaryInfC{$\neg P(f(y)) \vee \neg P(f(z))$}
    
  \end{prooftree}
  \caption{Пример не древовидного вывода в \emph{Conflict Resolution}}
  \label{fig:cr-example-1}
\end{figure}


Давайте рассмотрим несколько примеров.

\begin{example}
На рисунке \ref{fig:ecr-example-1} показан простой пример взаимодействия двух акторов со следующей специализацией: $\Psi_1 = \{A\}; \Psi_2 = \{B\}$. Акторы получают множество утверждений $\{A \vee B, \neg A, \neg B\}$ и показывают его противоречивость.
\end{example}

\begin{figure}
  \begin{prooftree}
    \AxiomC{$[P(x)^1]^1$}
	\AxiomC{$(\neg P(f(y)) \vee Q)^1$}
    \RightLabel{$\confl{1}{\{x \setminus f(y)\}}$}
    \BinaryInfC{$Q^1$}
    \RightLabel{$\kshare{1}{2}$}
	\UnaryInfC{$Q^2$}
  \end{prooftree}
  \caption{Пример преобразования \emph{Unit-Propagation}-1}
  \label{fig:ecr-hard-example-1}
\end{figure}

\begin{figure}
  \begin{prooftree}
    \AxiomC{$[P(x)^1]^1$}
    \AxiomC{$(\neg P(f(z)) \vee \neg Q)^1$}
    \RightLabel{$\confl{1}{\{x \setminus f(z)\}}$}
    \BinaryInfC{$(\neg Q)^1$}
    \RightLabel{$\kshare{1}{2}$}
    \UnaryInfC{$(\neg Q)^2$}
  \end{prooftree}
  \caption{Пример преобразования \emph{Unit-Propagation}-2}
  \label{fig:ecr-hard-example-2}
\end{figure}



%TODO: мб перенести этот пример в первую главу?
В связи с тем, что одно предположение может использоваться несколько раз, и вообще говоря доказательство представляет собой ациклический граф, а не дерево, в правиле \emph{CDCL} каждое предположение вырождается в клоз, состоящий из применения всех унификаций, использованых в доказательстве, к предположение. Следующий пример показывает это.
\begin{example}
\label{exmpl:cr-nonEPR}
На рисунке \ref{fig:cr-example-1} показан пример не древовидного доказательства. Здесь предположение $P(x)$ было использовано дважды, и поэтому существует несколько путей до противоречия в графе вывода.
\end{example}

\begin{figure}\begin{calculus}\centering

%%% Unit-Propagation inference rule
\textbf{Unit-Propagating Resolution:}
$$
\infer[\upr{i}{\sigma}]
	  {(\jmath_1^i \sigma) \vee \ldots \vee (\jmath_n^i \sigma) \vee 
       (\jmath^i \sigma) \vee (\ell^i \sigma)}
      {\jmath_1^i \vee \ell^i_1 & \ldots & 
       \jmath_n^i \vee \ell^i_n & 
       \jmath^i \vee 
       \overline{\ell'^i_1} \vee \ldots \vee \overline{\ell'^i_n} \vee \ell^i
      }
$$

\bigskip

%%% Conflict inference rule
\textbf{Conflict:}
$$
\infer[\confl{i}{\sigma}]
      {(\jmath_1^i \sigma) \vee (\jmath_2^i \sigma)}
      {(\jmath_1^i \vee \ell^i) & (\jmath_2^i \vee \overline{\ell'^i})}
$$ 

\bigskip

%%% CDCL inference rule
\textbf{Conflict-Driven Clause Learning:} 
$$
\infer[\cdcl{i}{p}]
      { \jmath \vee (\overline{\ell^i_1} \sigma^1_1 \vee \ldots \vee \overline{\ell^i_1} \sigma^1_{m_1}) \vee \ldots \vee (\overline{\ell^i_n} \sigma^n_1 \vee \ldots \vee \overline{\ell^i_n} \sigma^n_{m_n})
      }
	  { \infer*{\jmath}
               { \infer*[({\sigma'}_1^1,\ldots,{\sigma'}_{m'_1}^1)]
                      {}{\jmath^i_1} &  &
                 \infer*[({\sigma'}_1^1,\ldots,{\sigma'}_{m'_1}^1)]
                      {}{\jmath^i_m} &  &
                 \infer*[(\sigma_1^1,\ldots,\sigma_{m_1}^1)]
                      {}{[\ell^i_1]^{\dls{p}{1} } } &  &
                 \infer*[(\sigma_1^n,\ldots,\sigma_{m_n}^n)]
                      {}{[\ell^i_n]^{\dls{p}{n}}}
               }
      }
$$ 
где $\jmath = ({\jmath^i_1} \sigma'^1_1 \vee \ldots \vee \jmath^i_1 \sigma'^1_{m'_1}) \vee \ldots \vee ({\jmath^i_k} \sigma'^k_1 \vee \ldots \vee \jmath^i_k\sigma'^k_{m'_k})$

\bigskip

%%% Knowledge sharing inference rule
\textbf{Knowledge sharing:} 
$$
\infer[\kshare{i}{j}]
      {\phi^j}
	  {\phi^i}
$$

где $\phi^a$ --- клоз из актора $a$

\end{calculus}

\caption{Исчисление \emph{Expertised Conflict Resolution}}

\label{fig:ECR}
\end{figure}

\begin{figure}
  \begin{prooftree}
    \AxiomC{Вывод с рисунка \ref{fig:ecr-hard-example-1}}
    \UnaryInfC{$Q^2$}
    
    \AxiomC{Вывод с рисунка \ref{fig:ecr-hard-example-2}}
    \UnaryInfC{$(\neg Q)^2$}
    
    \RightLabel{$\confl{2}{}$}
    \BinaryInfC{$\bot$}
    
    \RightLabel{$\cdcl{2}{}$}
    \UnaryInfC{$(\neg P(f(y)) \vee \neg P(f(z)))^2$}
    \RightLabel{$\kshare{2}{1}$}
    \UnaryInfC{$(\neg P(f(y)) \vee \neg P(f(z)))^1$}
  \end{prooftree}
  \caption{Вывод в \emph{ECR}}
  \label{fig:ecr-hard-example}
\end{figure}

\section{Доказательство полноты по опровержению}

Из \cite{DBLP:journals/corr/SlaneyP16} мы знаем, что \emph{Conflict Resolution} полно относительно \emph{Resolution}, а значит достаточно показать, что \emph{Expertised Conflict Resolution} симулирует \emph{Conflict Resolution}. 


Далее мы покажем конструктивный способ как из вывода в \emph{CR} получить вывод в \emph{ECR}. 
\begin{example}
 Для вывода из примера \ref{exmpl:cr-nonEPR} и спецификации на акторах $\Psi_1 = \{P\}, \Psi_2 = \{Q\}$ алгоритм получит вывод, показанный на рисунке \ref{fig:ecr-hard-example}.
\end{example}

\subsection{Доказательство для части Бернайса-Шейфенкеля}
\label{sec:bsh-proof}
В этом подразделе далее везде будем считать, что наши формулы из класса Бернайа-Шейфенкеля. 

На данном этапе пример \ref{exmpl:cdcl-fail} не является проблемой, поскольку правило \emph{Knowledge sharing} позволяет нам обмениваться клозами между акторами, и никак не препятствует выводу \emph{CDCL}. Однако, для реализации это важный момент, который будет рассмотрен в третьей главе.

\begin{lemma}
\label{lem:upr}
Правило \emph{Unit-Propagation Resolution} из \emph{Conflict Resolution} можно выразить в исчислении \emph{Expertised Conflict Resolution}.
\end{lemma}
\begin{proof}
Обозначим за $L$ множество литералов $\{\ell_1, \ldots, \ell_n\}$ из правила \emph{Unit Propagation Resolution} на рисунке \ref{fig:CR}. $\ell$ --- выводимый литерал.
Доказательство в \emph{ECR} будем строить индукцией по размеру множества $L$:
\begin{itemize}[label=$\star$]
	\item \emph{База:} Существует такой актор $i$, что для любого литерала $\ell_k$ из $L$ верно, что актор $i$ специализируется на этом литерале, т.е. $\Psi_i(\ell_k)$. Если $\Psi_i(\ell)$, то в акторе $i$ можно применить правило \emph{Unit-Propagation Resolution}, описанное на рис. \ref{fig:ECR}, это правило даст нужное нам $\ell$. В противном случае правило \emph{Conflict} сократит все литералы, кроме выводимого $\ell$.
    \item \emph{Переход:} Так как $L$ --- конечное непустое множество, то можно выбрать произвольный литерал $\ell'$ из $L$, и актор $j$, который специализируется на этом литерале. Бывает два случая:
    \begin{enumerate}
    	\item Актор $j$ специализируется на выводимом литерале $\ell'$. Тогда в этом акторе можно применить правило \emph{Unit-Propagation Resolution}, которое даст $\ell \vee \jmath$, где $\jmath$ --- остаточный клоз, а множество $L$, соответствующее $\jmath$, будет содержать строго меньшее количество литералов.
        \item Актор $j$ не специализируется на $\ell'$. Тогда в этом акторе можно применить правило \emph{Conflict}, которое даст результат, аналогичный пункту (a).
    \end{enumerate} 
\end{itemize}
\end{proof}

\begin{lemma}
\label{lem:confl}
Правило \emph{Conflict} из \emph{Conflict Resolution} можно выразить в исчислении \emph{Expertised Conflict Resolution}.
\end{lemma}
\begin{proof}
В правиле \emph{Conflict} на рис. \ref{fig:CR} участвуют литералы $\ell$ и $\overline{\ell'}$. По свойству разбиения специализации по акторам, существует такой актор $j$, что $\Psi_j(\ell)$ и $\Psi_j(\overline{\ell'})$, а значит в нём правило \emph{Conflict} из \emph{EPR}, по $\ell^j\text{ и }(\overline{\ell'})^j$ выведет $\bot$.
\end{proof}

\begin{lemma}
\label{lem:cdcl}
Правило \emph{Conflict-Driven Clause Learning} из \emph{Conflict Resolution} можно выразить в исчислении \emph{Expertised Conflict Resolution}.
\end{lemma}
\begin{proof}
Правило \emph{Conflict-Driven Clause Learning} из исчисления \emph{CR} было описано на рисунке \ref{fig:CR}. Пусть в нём были сделаны предположения $D = \{\ell_1, \ldots, \ell_n\}$. Индукцией по размеру множества предположений мы можем назначить каждому литералу идентификатор актора, который специализируется на этом литерале, и в котором будет сделано данное предположение в доказательстве из исчисления \emph{ECR}. В \emph{CR} из множества $D$ выводится $\bot$, а значит по лемме \ref{lem:confl} существует вывод $\bot$ и в \emph{ECR}. После применения \ref{lem:confl} мы можем применить правило \emph{Conflict-Driven Clause Learning} из исчисления \emph{ECR}, и таким образом получить $\ell_1 \vee \ldots \vee \ell_n$.
\end{proof}

Таким образом, каждое правило из \emph{Conflict Resolution} мы выразили в \emph{Expertised Conflict Resolution}. Заметим, что мы сохранили структуру графа доказательства, а именно: не появилось новых литералов в процессе применения унификаций на путях. Теперь докажем основную теорему:

\begin{theorem}
\label{thm:completeness-bs}
Для формулы $\phi$ из класса Бернайса-Шейфенкеля, выводимой в \emph{Conflict Resolution}, существует номер актора $i$ и вывод формулы $\phi^i$ в \emph{Expertised Conflict Resolution}.
\end{theorem}
\begin{proof}
Заметим, что благодаря правилу \emph{Knowledge sharing}, если вершина выводима в акторе $i$, то она выводима в любом акторе $j$. Поэтому мы можем доказывать более сильное утверждение:
пусть $\psi$ --- вывод клоза $\phi$ из множества клозов $S$ в исчислении \emph{Conflict Resolution}. Покажем, что для любого $i$ существует вывод клоза $\phi^i$ из множества клозов $S$ в новом исчислении \emph{Expertised Conflict Resoution} индукцией по структуре доказательства в \emph{Conflict Resolution}.

\begin{itemize}[label=$\star$] 
\item \emph{База:} Единственная вершина $c$, содержащая формулу $\phi$ --- по начальному распределению она попадёт в какой-то актор $i$, а значит есть вывод $\phi^i$, следовательно есть вывод $\phi^j$ для произвольного $j$.
\item \emph{Переход:} Вершина $c$ получена из вершин $\{a_1, \ldots, a_n\}$, в которых выведены формулы $\{\phi_1, \ldots, \phi_n\}$ с помощью одного из правил: \emph{Unit-Propagation Resolution}, \emph{Conflict}, \emph{Conflict-Driven Clause Learning}. Тогда по индуктивному переходу для произвольного актора $i$ существует доказательство формул $\{\phi_1^i, \ldots, \phi_n^i\}$. Для каждого из правил мы доказали лемму, которая строит вывод этого правила в \emph{ECR}, зная доказательство в \emph{CR}: 
\begin{enumerate}
	\item Для правила \emph{Unit-Propagation Resolution} воспользуемся леммой \ref{lem:upr}.
    \item Для правила \emph{Conflict} воспользуемся леммой \ref{lem:confl}.
    \item Для правила \emph{Conflict-Driven Clause Learning} воспользуемся леммой \ref{lem:cdcl}.
\end{enumerate}
\end{itemize}
\end{proof}

Заметим, что нам необходимо было доказать полноту по опровержению, а мы построили для каждого доказательства из \emph{Conflict Resolution} соответствующее доказательство из \emph{Expertised Conflict Resolution}, тем самым доказали полноту.



% \subsection{Общий случай}


% В предыдущем разделе мы доказали полноту по опровержению для исчисления \emph{Expertised Conflict Resolution} в случае класса Бернайса-Шейфенкеля. Теперь давайте обобщим на литералы произвольной глубины.

% В общем случае 

% Для доказательства общего случая необходимо аккуратно показать, что у нас не появится новых литералов в результатирующих клозах.



% \begin{theorem}
% Пусть существует вывод клоза $\ell_1, \ldots, \ell_n$ в исчислении \emph{Conflict Resolution}.
% Тогда существует клоз $\ell'_1, \ldots, \ell'_m$, где $0 \leq m \leq n$, выводимый в исчислении \emph{Expertised Conflict Resolution}. Причём существует такая подстановка $\sigma$, что для любого $1 \leq i \leq m$ верно, что $\ell_i=\ell'_i\sigma$.
% \end{theorem}
% \begin{proof}
% Доказательство будем строить индукцией по структуре вывода в исчислении \emph{Conflict Resolution}.
% \begin{itemize}[label=$\star$]
% 	\item \emph{База:} Для формулы $\phi$ исходной вершины $c$, аналогично базе доказательства \ref{thm:completeness-bs}, существует доказательство формулы $\phi^i$.
%     \item \emph{Переход:} 
%       \begin{enumerate}
%       	\item \emph{Unit-Propagation Resolution.} Из рисунка \ref{fig:CR} видно, что в этом правиле используются следующие клозы: $\{\ell_1, \ldots, \ell_n, \overline{\ell'_1} \vee \ldots \overline{\ell'_n} \vee \ell\}$. Рассмотрим три случая: 
%         \begin{enumerate}
%         	\item Хотя бы для одного из клозов из $\{\ell_1, \ldots, \ell_n\}$ в \emph{ECR} вывелось доказательство $\bot^i$. Тогда этот $\bot^i$ подходит и для клоза $\ell$.
%             \item Для каждого $\ell_j$ построилось доказательство для формулы ${\ell'}_j^i$, причём существует подстановка ${\ell'}_j^i\sigma'=\ell_j$. А также для клоза $\overline{\ell'_1} \vee \ldots \overline{\ell'_n} \vee \ell$  построилось доказательство для формулы $\overline{{\ell''}_1} \vee \ldots \overline{{\ell''}_k} \vee \ell'$. В данном случае мы можем сослаться на лемму \ref{lem:upr} и построить вывод формулы ${\ell'}^i$ для некоторого актора $i$.
%         \end{enumerate}
%         \item \emph{Conflict.} Как показано на рисунке \ref{fig:CR}, в данном правиле используются $\ell$ и $\ell'$. Значит по предположению индукции бывает два случая:
%         \begin{enumerate}
%         	\item Хотя бы в одном из случаев $\ell, \ell'$ выводится $\bot$. Этот случай очевиден.
%             \item Существует вывод для $\ell^i, (\ell')^i$, таких что $\ell^i\sigma'=\ell$ и $(\ell')^i\sigma'=\ell'$. Из \emph{CR} следует, что существует унификация для правила вывода $\mathbf{c}(\sigma)$ на литераллах $\ell, \ell'$. Тогда $\mathbf{c_i}(\sigma \cdot \sigma')$ -- правило вывода \emph{Conflict} в \emph{ECR} для литераллов $\ell^i, (\ell')^i$.
%         \end{enumerate}
%         \item \emph{Conflict-Driven Clause Learning.} Как показано на рисунке \ref{fig:CR}, в данном правиле используются предположения $\{[\ell_1], \ldots, [\ell_n]\}$, по лемме, аналогичной лемме \ref{lem:cdcl}, мы получим клоз $\jmath \vee ((\ell^i_1\sigma^1_1) \vee \ldots \vee (\ell^i_{m_1}\sigma^1_{m_1})) \vee ((\ell^i_n\sigma^n_1) \vee \ldots \vee (\ell^i_{m_n}\sigma^n_{m_n}))$. 
%       \end{enumerate}
% \end{itemize}
% \end{proof}

% \begin{theorem}
% Для любой формулы $\phi$, выводимой в \emph{Conflict Resolution}, существует номер актора $i$ и вывод формулы $\phi^i$ в \emph{Expertised Conflict Resolution}.
% \end{theorem}
% \begin{proof}
% Доказательство будем строить индукцией по максимальной глубине литерала в клозе.
% \begin{itemize}[label=$\star$]
% 	\item \emph{База:} Все литералы в клозе имеют глубину $0$, иначе говоря предикаты не могут иметь аргументов. Этот случай был доказан в разделе \ref{sec:bsh-proof}.
%     \item \emph{Переход:} 
% \end{itemize}
% \end{proof}

\section{Динамическое изменение специализации}
\label{sec:onlineChanging}
\begin{figure}\begin{calculus}\centering

\textbf{Add specialization:}
$$
\infer[\mathbf{as_i(P)}]
      {\Psi_i \cup \{P\}}
	  {\Psi_i}
$$

\bigskip

\textbf{Remove specialization:}
$$
\infer[\mathbf{rs_i(P)}]
      {\Psi_i \setminus \{P\}}
      {\Psi_i}
$$ 
\end{calculus}
\caption{Обобщение \emph{Expertised Conflict Resolution} на динамическую специализацию}
\label{fig:onlineC}
\end{figure}

Обратим внимание, что у акторов есть возможность менять функцию \emph{receive}, а также есть внутреннее состояние, в котором могут быть изменяемые множества. Рассмотрим случай, когда специализация характеризуется множеством предикатов, порождающим множество литералов. На рисунке \ref{fig:onlineC} показаны два новых правила вывода, здесь: 
\begin{enumerate}
	\item Правило $\mathbf{as_i(P)}$ добавляет в специализацию актора $i$ предикат $P$.
    \item Правило $\mathbf{rs_i(P)}$ удаляет из специализации актора $i$ предикат $P$.
\end{enumerate}

Данная интерпретация интересна тем, что позволяет нашей акторной модели динамически распределять нагрузку на акторы. В частности это может быть одним из решений проблемы, описанной в разделе \ref{sec:impl-fail}.

Однако, в реализации системы автоматического доказательства теорем, основанной на динамической специалиации, сложным вопросом является проблема выбора моментов времени, когда нужно эти правила применять, и какие именно предикаты следует добавить или убрать. Для дальнейшего обсуждения предлагается обратить внимание на следующие подходы:
\begin{enumerate}
	\item Изменять специализацию внутри актора, если происходит \emph{голодание}, то есть за последние $K$ шагов актор не смог вывести ни одного нового клоза. 
    \item Поддерживать акторы специального вида, так называемые \emph{Распределители}. Такие акторы будут собирать статистику и использовать обучение с подкреплением для правильного распределения нагрузки.
\end{enumerate}


\section{Выводы ко второй главе}

\begin{enumerate}
	\item Введено новое исчисление \emph{Expertised Conflict Resolution}, которое обобщает \emph{Conflict Resolution} на модель акторов.
    \item Доказана полнота по опровержению исчисления \emph{ECR}.
    \item Предложено обобщение на динамическую специализацию.
\end{enumerate}