\documentclass[times,specification,annotation]{itmo-student-thesis}

\usepackage{icomma}

\usepackage{tikz}
\usepackage{systeme}
\usetikzlibrary{arrows}
\usepackage{filecontents}
\usepackage{fancybox}
\usepackage{array}
\usepackage{proof}
\usepackage{fontspec}
\usepackage{bussproofs}

\usetikzlibrary{graphs}

\graphicspath{ {images/} }



%% box для правил вывода
\newenvironment{calculus}
{\begin{center}\begin{Sbox}\begin{minipage}[b]{0.98\textwidth}}
{\end{minipage}\end{Sbox}\fbox{\TheSbox}\end{center}}

%% правила вывода из оригинальной статьи
\newcommand{\upr}[2]{\mathbf{u}_{#1}(#2)}
\newcommand{\confl}[2]{\mathbf{c}_{#1}(#2)}
\newcommand{\cdcl}[2]{\mathbf{cl}_{#1}^{#2}}

%% новые правила
\newcommand{\kshare}[2]{\mathbf{ks}^{#1}_{#2}}

\newcommand{\insp}[2]{\emph{$\Psi$}_{#1}(#2)}

\newcommand{\dls}[2]{\begin{tiny}{#1}_{#2}\end{tiny}}

%% Указываем файл с библиографией.
\addbibresource{bachelor-thesis.bib}

\begin{document}

\studygroup{M3439}
\title{Параллелизация системы автоматического доказательства теорем в теории первого порядка}
\author{Подтелкин Владислав Евгеньевич}{Подтелкин В.Е.}
\supervisor{Штукенберг Дмитрий Григорьевич}{Штукенберг Д.Г.}{тьютор каф. КТ}{}
\publishyear{2018}
\startdate{01}{сентября}{2017}
\finishdate{31}{мая}{2018}

\defencedate{10}{июня}{2018}

\secretary{Павлова О.Н.}

%% Задание
%%% Техническое задание и исходные данные к работе
\technicalspec{Требуется обобщить исчисление Conflict Resolution на модель Акторов. Необходимо реализовать алгоритм на основе обобщённого исчисления и протестировать его на задачах из библиотеки TPTP.}

%%% Содержание выпускной квалификационной работы (перечень подлежащих разработке вопросов)
\plannedcontents{
  \begin{enumerate}
    \item Описание исчисления Conflict Resolution в модели Акторов
    \item Реализация алгоритма, основанного на новом исчислении
    \item Проведение экспериментов и сравнение результатов с существующими решениями 
  \end{enumerate}
}

%%% Исходные материалы и пособия 
\plannedsources{
  \begin{enumerate}
    \item Slaney J., Woltzenlogel Paleo B. Conflict Resolution: a First-Order
      Resolution Calculus with Decision Literals and Conflict-Driven Clause
      Learning
      
    \item Itegulov D., Slaney J., Woltzenlogel Paleo B. Scavenger 0.1: A Theorem Prover Based on Conflict Resolution
    
    \item Исходный код программного средства Scavenger
  \end{enumerate}
}

%%% Календарный план
\addstage{Ознакомление с исчислением Conflict Resolution}{01.10.2017}
\addstage{Ознакомление с кодовой базой Scavenger}{15.10.2017}
\addstage{Построение общих концептов работы алгоритма}{01.12.2017}
\addstage{Реализация различных версий предложенного алгоритма}{01.02.2018}
\addstage{Экспериментальное исследование алгоритма и сравнение с другими системами автоматических доказательств}{01.03.2018}
\addstage{Написание пояснительной записки}{30.05.2018}

%%% Цель исследования
\researchaim{Обобщить исчисление Conflict Resolution на модель Акторов. Разработать алгоритм на основе нового исчисления.}

%%% Задачи, решаемые в ВКР
\researchtargets{
  \begin{enumerate}
      \item Описание преимуществ и недостатков акторной модели по сравнению с другими подходами
      \item Обобщение исчисления на Акторную модель
      \item Разработка алгоритма на основе нового исчисления
      \item Доказательства полноты относительно опровержения для нового исчисления и для алгоритма
      \item Реализация алгоритма на языке программирования Scala
  \end{enumerate}
}

%%% Использование современных пакетов компьютерных программ и технологий
\advancedtechnologyusage{Был использован язык \emph{Scala}, фреймворки \emph{akka-actors}, \emph{scalatest}, система контроля версий git, система компьютерной вёрстки \emph{LaTeX}}

%%% Краткая характеристика полученных результатов 
\researchsummary{Разработано новое исчисление и написан распределённый алгоритм, на основе этого исчисления.}

%%% Гранты, полученные при выполнении работы 
\researchfunding{
  Работа была частично проспонсирована программой Google Summer of Code.
}

%%% Наличие публикаций и выступлений на конференциях по теме выпускной работы
\researchpublications{Публикации и выступления отсутствуют.}

%% Эта команда генерирует титульный лист и аннотацию.
\maketitle{Бакалавр}

% Оглавление
\tableofcontents
                                                                                                                            
\startprefacepage

%Актуальность
Система Автоматического Доказательства Теорем (далее САДТ) -- популярная область математической логики, дающая обширные практические результаты. С помощью САДТ можно, в частности, помогать в доказательстве теорем из различных областей математики (например интерактивное программное средство доказательства теорем \emph{Coq}), тестировать интегральные схемы и верифицировать программный код (например \cite{Detlefs:2005:STP:1066100.1066102}, \cite{zap-automated-theorem-proving-for-software-analysis}) тестировать криптографические протоколы (например \cite{DBLP:journals/corr/MoranW17}).


Существует и ежегодно дополняется архив задач \emph{TPTP} \cite{Sutcliffe2009}. На нём можно найти примеры, затрагивающие огромное количество областей, представленные в стандартном для САДТ виде.


%Цель
Целью данной работы является обобщение уже существующего исчисления \emph{Conflict Resolution} \cite{DBLP:journals/corr/SlaneyP16} на модель акторов. Рассмотрение различных подходов к оптимизации нового исчисления, а именно: динамическое распределение мощностей и добавление дополнительных ограничений на правила вывода, связанных с коммуникацией акторов.


%Новизна
Несмотря на большое количество плюсов акторной модели, ни один из существующих на сегодняшний день подходов для параллелизации систем автоматического доказательства теорем не использует её. Стоит, впрочем, упомянуть, что в обзорной работе \cite{Bonacina2018} был упомянут подход, использующий диффундирующие вычисления. 
%% TODO: ПОЯСНИТЬ чем диффундирующие вычисления близки к Акторам.

%Практическое значение
В акторной модели общение между вычислительными процессами(акторами) происходит при помощи асинхронного обмена сообщениями, которые посылаются от актора к актору. Отсутствие понятия \emph{общей памяти} в акторной модели позволяет не только распараллелить вычислительные процессы, но и предоставить возможность распределённых вычислений (например, при помощи фреймворка \emph{akka-remote}).


% Краткий обзор струткруы работы
%% глава 1
В главе \ref{sec:chap1} приводятся основы логики первого порядка, формулируется решаемая задача. Также будет введена базовая версия решения задачи, основанная на исчислении \emph{Conflict Resolution}. 


%% глава 2
В главе \ref{sec:chap2} будет представлено обобщение исчисления \emph{Conflict Resolution} на акторную модель, доказана полнота по опровержению полученного исчисления. Будет предложено введение дополнительных ограничений на правила вывода с соответствующими доказательствами полноты относительно опровержения. Также будет построено исчисление с динамическим изменением специализации акторов.

%% глава 3
В главе \ref{sec:chap3} будут кратко описаны детали программной реализации, дана оценка производительности на наборе задач \emph{TPTP}, представлены графики для сравнения новой программы с конкурирующими средствами.
\chapter{Введение в область}
\startrelatedwork
\label{sec:chap1}

\section{Исчисление предикатов первого порядка}

Далее мы опишем нотацию, предложенную Alexander Leitsch \cite{Leitsch:1997:RC:260906}.

\subsection{Определение исчисления}

Далее будем считать, что нам даны:
\begin{enumerate}
	\item $V$    --- счётное множество названий переменных
    \item $CS$   --- множество константных символов
    \item $FS_i$ --- множество $i$-нарных функциональных символов, где $i \in \mathbb{N}$. \\
    За $FS$ обозначим $\bigcup\limits_{i=1}^{\infty} FS_{i}$
    \item $PS_i$ --- множество $i$-нарных предикатных символов, где $i \in \mathbb{N}$. \\
    За $PS$ обозначим $\bigcup\limits_{i=1}^{\infty} PS_{i}$
    \item Логические связки --- символы $\vee, \wedge, \rightarrow, \neg$
    \item Кванторы --- символы $\forall, \exists$
\end{enumerate}



\begin{definition}
  \emph{Множество термов $T$.} Минимальное по включению множество, определяемое индуктивно:
  \begin{enumerate}
  	\item $V \subseteq T$
    \item $CS \subseteq T$
    \item Если $t_1, t_2, \ldots, t_k \in T; f \in FS_k$, тогда $f(t_1, t_2, \ldots, t_k) \in T$
  \end{enumerate}
  будем называть \emph{множеством термов}.
\end{definition}

\begin{definition}
  \emph{Атомарная формула.} Пусть $t_1, t_2, \ldots, t_k \in T; P \in PS_k$, тогда формула $P(t_1, t_2, \ldots, t_k)$ называется атомарной.
\end{definition}

Множество атомарных формул обозначим за $AT$. 

\begin{definition}
  \emph{Литерал.} Пусть $A$ -- атомарная формула, тогда $A, \neg A$ -- литералы.
\end{definition}

Множество литералов обозначим за $LIT$

Примеры литералов:
\begin{example}
Рассмотрим следующий литерал
$\neg ODD(mul(2,x))$.  
\begin{enumerate}
	\item $ODD$ --- предикат
    \item $mul$ --- функциональный символ
    \item $2$ --- константный символ
    \item $x$ --- переменная
\end{enumerate}
Если мы назначим естественную оценку этому литералу, то он будет отвечать на вопрос "правда ли, что для любого $x$, $2*x$ -- чётно". 
\end{example}


\begin{definition}
  \emph{Множество формул исчесления предикатов $PL$.} минимальное по включению множество, определяемое индуктивно:
  \begin{enumerate}
  	\item $LIT \subseteq PL$
    \item Пусть $A, B \in PL$, тогда $A \to B, A \vee B, A \wedge B \in PL$ 
    \item Пусть $A \in PL, x \in V, (\forall x) \notin A, (\exists x) \notin A$, тогда $(\forall x)A, (\exists x)A \in PL$
  \end{enumerate}
\end{definition}

Иначе говоря, формулы исчисления предикатов -- это литералы, связанные логическими операциями и кванторами.

\subsection{Интерпретация}

\begin{definition}
  \emph{Свободное вхождение.} Определим индуктивно: 
  \begin{enumerate}
  	\item Пусть формула $A \in LIT$. Тогда $x$ входит свободно, если $x$ встречается в $A$.
    \item Пусть формула $A = B \odot C$, где $\odot \in \{ \to, \vee, \wedge \}$. Тогда $x$ входит свободно в $A$, если $x$ входит свободно в $B$, либо $C$.
    \item Пусть формула $A = (Qy)B$, где $Q \in \{ \exists, \forall \}$. Тогда $x$ входит свободно в $A$, если $x$ входит свободно в $B$ и $y \neq x$
  \end{enumerate}
\end{definition}

\begin{definition}
  \emph{Замкнутое вхождение.} $x$ входит замкнуто в формулу $A$, если у $A$ существует подформула $B$ вида $(Qx)B'$, где $Q \in \{ \exists, \forall \}$ и $x$ входит свободно в $B'$
\end{definition}

\begin{definition}
  \emph{Открытая и замкнутые формулы.}
    \begin{enumerate}
      \item Если у формулы нет свободных вхождений переменных, то она называется \textit{замкнутой}.
      \item Если у формулы нет замкнутых вхождений переменных, то она называется \textit{открытой}.
    \end{enumerate}
\end{definition}

\begin{definition}
  \emph{Интерпретация формулы.} Интерпретацией формулы $F$ из $PL$ назовём тройку $\Gamma = (D, \Phi, I)$, для которой верно следующее:
    \begin{enumerate}
		\item $D$ --- непустое множество, которое называется доменом $\Gamma$
        \item $\Phi$ --- отображение из $CS(F) \cup FS(F) \cup PS(F)$, заданное так:
          \begin{enumerate}
              \item $\Phi(c) \in D$, если $c \in CS(F)$
              \item Если $f \in FS_n(F)$, то $\Phi(f) : D^n \rightarrow D$
              \item Если $P \in PS_n(F)$, то $\Phi(P) : D^n \rightarrow \{ 0, 1 \}$
          \end{enumerate}
        \item $I$ -- фукнция из $V$ в $D$, которая задаёт \textit{назначение переменных}
    \end{enumerate}

\end{definition}


\begin{definition}
  \emph{Эквивалентность интерпретаций.} Две интерпретации $\Gamma$ и $\Delta$ называются эквивалентными по модулю $x_1, x_2, \ldots, x_k$, если существуют такие $D, \Phi, I_1, I_2$, что $\Gamma = {(D, \Phi, I_1)}$, $\Delta = {(D, \Phi, I_2)}$, а также верно $v \notin \{x_1, x_2, \ldots, x_k\} \Rightarrow I_1(v) = I_2(v)$. Далее будем писать $\Gamma \sim_x \Delta$.
\end{definition}

\begin{definition}
  \emph{Оценка исчисления предикатов.} Введём оценку $\nu_{\Gamma}$ как функцию из формулы в $\{0, 1\}$, которая будет определяться индукцией по структуре:
  \begin{enumerate}
  	\item Если формула $A$ -- атомарная, т.е. имеет вид $P(t_1, t_2, \ldots, t_k)$, то $\nu_{\Gamma}(A) = \Phi(P)(\nu_{\Gamma}(t_1), \nu_{\Gamma}(t_2), \ldots, \nu_{\Gamma}(t_k))$
  	\item \begin{enumerate}
    		\item $\nu_\Gamma(A \vee B) = \nu_\Gamma(A) \vee \nu_\Gamma(B)$
    		\item $\nu_\Gamma(A \wedge B) = \nu_\Gamma(A) \wedge \nu_\Gamma(B)$
    		\item $\nu_\Gamma(A \rightarrow B) = \nu_\Gamma(A) \rightarrow \nu_\Gamma(B)$
    		\item $\nu_\Gamma(\neg A) = \neg(\nu_\Gamma(A))$
    	  \end{enumerate}
    \item $\nu_\Gamma((\forall x)A) = 1$, если для любой $\Delta$ верно $\Gamma \sim_x \Delta \iff \nu_\Delta(A) = 1$ 

 \item $\nu_\Gamma((\exists x)A) = 1$, если существует такая $\Delta$, что $\Gamma \sim_x \Delta$ и $\nu_\Delta(A) = 1$
  \end{enumerate}
\end{definition}

Будем говорить, что 
\begin{enumerate}
	\item Интерпретация $\Gamma$ \textit{верифицирует} формулу $A$, если $\nu_{\Gamma}(A) = 1$.
	\item Интерпретация $\Gamma$ \textit{опровергает} формулу $A$, если $\nu_{\Gamma}(A) = 0$.
\end{enumerate}

\subsection{Удовлетворимость и корректность}

\begin{definition}
  \emph{Модель.} Пусть дана формула $A$, содержащая свободные вхождения переменных $x_1, x_2, \ldots, x_k$, тогда интерпретация $\Gamma$ является \textit{моделью} для $A$, если для любой интерпретации $\Delta \sim_x \Gamma$ верно, что $\nu_{\Delta}(A) = 1$. Если $A$ -- замкнутая формула, то $\Gamma$ является моделью тогда и только тогда, когда $\Gamma$ верифицирует $A$.
\end{definition}

\begin{definition}
  \emph{Удовлетворимость и корректность.} 
  \begin{enumerate}
  	\item Формула $F$ называется \textit{удовлетворимой}, если для неё существует модель.
  	\item Формула $F$ называется \textit{корректной}, если любая интерпретация $\Gamma$ для этой формулы является моделью.
  	\item Формулы $F$ и $G$ \textit{логически эквивалентны}, если у них одинаковое множество моделей. Обозначение $F \sim G$
  	\item Будем писать $F \sim_{sat} G$, если $F \text{ -- удовлетворима} \iff G \text{ -- удовлетворима}$
  \end{enumerate}
\end{definition}

\begin{definition}
  \emph{Клоз.} Пусть $L_1, L_2, \ldots, L_k$ -- литералы, тогда формулу вида $L_1 \vee L_2 \vee \dots \vee L_k$ будем называть \emph{клозом}
\end{definition}

\begin{definition}
  \emph{Конъюктивная нормальная форма.} Пусть $C_1, C_2, \ldots, C_m$ --- \emph{клозы}, тогда формулу вида $C_1 \wedge C_2 \wedge \dots \wedge C_m$ находится в Конъюктивно Нормальной Форме(\emph{КНФ}).
\end{definition}

Заметим, что для любой формулы из \emph{Исчисления предикатов первого порядка}, можно найти эквивалентную ей формулу в \emph{КНФ}. Это делается при помощи \textit{Клозификации} \cite{clausification}. Стоит, впрочем, упомянуть, что размер эквивалентной формулы имеет экспоненциальный размер относительно оригинальной формулы исчисления.

Заметим также, что формула $A$ корректна тогда и только тогда, когда $\neg A$ не удовлетворима. А из наблюдений выше вопрос об удовлетворимости произвольной формулы эквивалентен вопросу об удовлетворимости множества клозов. 

В отличие от \textit{Исчисления Высказываний}, в \textit{Исчислении Предикатов} мы не можем проверить все интерпретации, так как, вообще говоря, их бесконечно много.\\

\section{Резолюции}

\subsection{Резолюционное исчисление}
\texttt{ввести CDCL}

\subsection{Conflict Resolution}
\label{sec:CR}
Исчисление \emph{Conflict Resolution} является обобщением классических идей метода из Резолюционного исчисления \emph{CDCL} на теореию первого порядка. Обзор других подходов к решению данной задачи можно найти в \cite{DBLP:journals/corr/SlaneyP16}. 

На рисунке \ref{fig:CR} описаны правила вывода для исчисления \emph{Сonflict Resolution} \cite{DBLP:journals/corr/SlaneyP16}. Здесь за $\ell$ обозначаются литералы. Особенностью этого исчисления является возможность делать предположения, которые на рисунке берутся в квадратные скобки.

\begin{figure}
\begin{calculus}
\centering
\textbf{Unit-Propagating Resolution:}
$$
\infer[\upr{}{\sigma}]
	  {\ell~\sigma}
      {\ell_1 & \ldots & \ell_n & \overline{\ell'_1} \vee \ldots \vee \overline{\ell'_n} \vee \ell}
$$

где $\sigma$ --- унификатор $\ell_k$ и $\ell'_k$ для всех $k \in \{1, \ldots, n \}$.

\bigskip

\textbf{Conflict:}
$$
\infer[\confl{}{\sigma}]
      {\bot}
      {\ell & \overline{\ell'}}
$$

где $\sigma$ --- унификатор $\ell$ and $\ell'$.

\bigskip

\textbf{Conflict-Driven Clause Learning:}
$$
\infer[\cdcl{}{i}]
      {(\overline{\ell_1} \sigma^1_1 \vee \ldots \vee \overline{\ell_1} \sigma^1_{m_1}) \vee \ldots \vee (\overline{\ell_n} \sigma^n_1 \vee \ldots \vee \overline{\ell_n} \sigma^n_{m_n})
      }
	  {\infer*{\bot}{\infer*[(\sigma_1^1,\ldots,\sigma_{m_1}^1)]{}{[\ell_1]^{\dls{i}{1}}} &  & \infer*[(\sigma_1^n,\ldots,\sigma_{m_n}^n)]{}{[\ell_n]^{\dls{i}{n}}}}
      }
$$

где $\sigma^k_j$ (для $1 \leq k \leq n$ и $1 \leq j \leq m_k$) -- это \\
композиция всех подстановок, использованных на $j$-ом пути \footnote{Так как
  доказательство -- это ориентированный ациклический граф, который возможно не
  является деревом, то может существовать несколько путей, соединяющих $\ell_k$ и
  $\bot$ в доказательстве.} от $\ell_k$ до $\bot$.

\end{calculus}
\caption{Исчисление \emph{Conflict Resolution}}
\label{fig:CR}
\end{figure}


\section{Акторы}

Модель акторов, впервые была упомянута в \cite{Hewitt:1973:UMA:1624775.1624804}. В связи с тем, что большинство формальных описаний акторной модели используют последовательные
\begin{definition}
  \emph{Актор}. Вершина графа $x$, для которой возможны следующие операции:
  \begin{enumerate}
  	\item $receive(msg)$ --- принимает сообщение $msg$ от другого актора и выполняет функцию $fun(msg)$
    \item $send(actorId, msg)$ --- в процессе выполнения функции $fun$ из предыдущего пунтка, актор может послать сообщение $msg$ другому актору $actorID$
  \end{enumerate}
\end{definition}


\section{Примеры}


\chapter{Постановка задачи}
\startrelatedwork
\label{sec:chap2}


\section{Мотивация}
Систему для автоматического доказательства теорем можно рассматривать как совокупность небольших подсистем (акторов). В этой системе каждый актор:
\begin{enumerate}
	\item Получает некоторые формулы от других акторов
    \item Выделяет из них подформулы, на которых он специализируется (например только часть с предикатами из заданного множества)
    \item Делает выводы из формул по своей специализации
    \item сообщает эту информацию другим акторам (следующему по кругу, либо всем, либо по какому-нибудь заданному алгоритму)
\end{enumerate} \par
Интуитивная мотивация такого подхода происходит из социиологии, где люди не решают сложные задачи поодиночке. Например, если у меня болит горло, то я сделаю некоторые допущения о своём состоянии и обращусь к доктору. Таким образом, у каждой сущности есть небольшое подмножество задач, соответсвующих специализации этой сущности. А решение задачи происходит в процессе коммуникации сущностей.

\section{Формальное построение модели}

\subsection{Описание акторов}

Так как у каждого актора есть внутреннее состояние, то мы можем определять функции и предикаты, зависящие от номера актора. Давайте введём функцию, котороая будет распределять литералы по акторам, то есть отвечать на вопрос "Работает ли актор $A$ с литералом $\ell$".

\begin{definition}
  \emph{Специализация.} Пусть $A = \{A_1, A_2, \ldots, A_N\}$ --- множество из $N$ акторов. 
  Тогда семейство предикатов $\{\Psi_{i}\}_{i \in \{1,\ldots,N\}}$, заданных на множестве литералов $LIT$, 
  называется \emph{специализацией} актора. \par
  
  Будем говорить, что актор $i$ \emph{специализируется} на литерале $\ell$, если верно $\insp{i}{\ell}$.
\end{definition}

На данном этапе будем считать, что \emph{специализация} задана заранее и не меняется в процессе доказательства. Однако, в разделе \ref{sec:onlineChanging} мы рассмотрим вариант с двумя дополнительными правилами вывода, который подразумевает изменение \emph{специализации} в процессе доказательства.

\begin{definition}
	\emph{Остаточный клоз.} Пусть $C = \ell_1 \vee \ldots \vee \ell_n$ -- клоз. Тогда $rem_a(C) = \ell_{i_1} \vee \ldots \vee \ell_{i_k}$ -- остаточный клоз относительно актора $a$, если $\Psi_a(\ell_i) \iff i \in \{i_1, \ldots, i_k\}$
\end{definition}

%% TODO: изменение этой функции в процессе доказательства, т.к. у акторов есть функция become, которая изменяет поведение актора.

\subsection{Описание Expertised Conflict Resolution исчисления}

На рисунке \ref{fig:ECR} показаны правила для обобщённого на распределённый случай исчисления \emph{Conflict Resolution}, которое было описано в секции \ref{sec:CR}. \par

На правила вывода накладываются следующие дополнительные условия:
\begin{enumerate}
	\item \emph{Unit-Propagation Resolution.}
    \begin{itemize}
    	\item ${\ell^i_1, \ldots, \ell^i_n, \ell'^i_1, \ldots, \ell'^i_n, \ell}$ -- литералы из специализации актора $i$, т.е. для любого $k$ верно, что $\insp{i}{\ell^i_k}$, $\insp{i}{\ell'^i_k}$, $\insp{i}{\ell^i}$
        \item $\sigma$ -- унификатор для $\ell^i_k$ и $\ell'^i_k$ при всех $k \in \{1, \ldots, n \}$ 
        \item $\jmath^i, \jmath_1^i, \ldots, \jmath_n^i$ -- остаточные клозы соответствующих высказываний из предпосылки.
    \end{itemize}
        
	\item \emph{Cofnlict.}
    \begin{itemize}
        \item $\ell^i, \ell'^i$ -- литералы из актора $i$, т.е. верно $\insp{i}{\ell^i_k}$ и $\insp{i}{\ell'^i_k}$
    	\item $\sigma$ -- унификатор $\ell$ и $\ell'$
        \item $\jmath_1^i, \jmath_2^i$ -- остаточные клозы соответствующих высказываний из предпосылки.
    \end{itemize}
        
    \item \emph{Conflict-Driven Clause Learning.}
    \begin{itemize}
    	\item ${\ell^i_1, \ldots, \ell^i_n, \ell'^i_1, \ldots, \ell'^i_n, \ell}$ -- литералы из актора $i$, т.е. для любого $k$ верно, что $\insp{i}{\ell^i_k}$, $\insp{i}{\ell'^i_k}$, $\insp{i}{\ell^i}$
        
		\item $\overline{\insp{i}{\jmath^i_k}}$ для всех $k \in \{1, \ldots, m\}$ 

		\item $\sigma^k_j$ (для $1 \leq k \leq n$ и $1 \leq j \leq m_k$) -- это
композиция всех подстановок, использованных на $j$-ом пути \footnote{Так как
  доказательство -- это ориентированный ациклический граф, который возможно не
  является деревом, то может существовать несколько путей, соединяющих $\ell^i_k$ и
  $\jmath$ в доказательстве.} от $\ell^i_k$ до $\jmath = \jmath^i_1 \vee \ldots \vee \jmath^i_m$
  
  		\item  $\sigma'^k_j$ (для $1 \leq k \leq m$ и $1 \leq j \leq m'_k$) -- это
композиция всех подстановок, использованных на $j$-ом пути от $\jmath^i_k$ до $\jmath$
    \end{itemize}
    
    \item \emph{Konwledge sharing.}
    На данном этапе не будем описывать дополнительных условий для этого правила вывода, однако будут рассмотрены идеи ограничений, которые помогут нам ускорить реализацию данного алгоритма.
\end{enumerate}

В отличие от классического исчисления \emph{Conflict Resolution}, здесь каждому клозу приписывается ещё и номер актора, в котором на данный момент этот клоз был выведен. Правило \emph{Knowledge sharing} позволяет изменять этот номер. В предпосылках у старых правил \emph{Unit-Propagation}, \emph{Conflict} и \emph{Conflict-Driven Clause Learning} теперь вместо чистых литералов фигурируют целые клозы, которые, однако, имеют специальный вид: $\ell \vee \jmath$, где $\ell$ -- литерал, а $\jmath$ -- остаточный клоз соответствующего актора.



\begin{figure}
  \begin{prooftree}
    \AxiomC{$(A \vee B)^1$}
    \AxiomC{$\neg A^1$}
    \RightLabel{$\confl{1}{}$}
    \BinaryInfC{$B^1$}
    \RightLabel{$\kshare{1}{2}$}
    \UnaryInfC{$B^2$}
    \AxiomC{$\neg B^2$}
    \RightLabel{$\confl{2}{}$}
    \BinaryInfC{$\bot$}
  \end{prooftree}
  \caption{Пример взаимодействия акторов}
  \label{fig:ecr-example-1}
\end{figure}

\begin{figure}
  \begin{prooftree}
    \AxiomC{}
    \RightLabel{$\mathbf{d}$}
    \UnaryInfC{$[P(x)]^1$}
    
	\AxiomC{$\neg P(f(y)) \vee Q$}
    \RightLabel{$\upr{}{\{x \setminus f(y)\}}$}
    \BinaryInfC{$Q$}
    
    \AxiomC{$[P(x)]^1$}
    \AxiomC{$\neg P(f(z)) \vee \neg Q$}
    \RightLabel{$\upr{}{\{x \setminus f(z)\}}$}
    \BinaryInfC{$\neg Q$}
    
    \RightLabel{$\confl{}{}$}
    \BinaryInfC{$\bot$}
    
    \RightLabel{$\cdcl{}{}$}
    \UnaryInfC{$\neg P(f(y)) \vee \neg P(f(z))$}
    
  \end{prooftree}
  \caption{Пример не древовидного вывода в \emph{Conflict Resolution}}
  \label{fig:cr-example-1}
\end{figure}


Давайте рассмотрим несколько примеров.

\begin{example}
На рисунке \ref{fig:ecr-example-1} показан простой пример взаимодействия двух акторов со следующей специализацией: $\Psi_1 = \{A\}; \Psi_2 = \{B\}$. Акторы получают множество утверждений $\{A \vee B, \neg A, \neg B\}$ и показывают его противоречивость.
\end{example}

%TODO: мб перенести этот пример в первую главу?
В связи с тем, что одно предположение может использоваться несколько раз, и вообще говоря доказательство представляет собой ациклический граф, а не дерево, в правиле \emph{CDCL} каждое предположение вырождается в клоз, состоящий из применения всех унификаций, использованых в доказательстве, к предположение. Следующий пример показывает это.
\begin{example}
На рисунке \ref{fig:cr-example-1} показан пример не древовидного доказательства. Здесь предположение $P(x)$ было использовано дважды, и поэтому существует несколько путей до противоречия в графе вывода.
\end{example}


\begin{figure}\begin{calculus}\centering

%%% Unit-Propagation inference rule
\textbf{Unit-Propagating Resolution:}
$$
\infer[\upr{i}{\sigma}]
	  {(\jmath_1^i \sigma) \vee \ldots \vee (\jmath_n^i \sigma) \vee 
       (\jmath^i \sigma) \vee (\ell^i \sigma)}
      {\jmath_1^i \vee \ell^i_1 & \ldots & 
       \jmath_n^i \vee \ell^i_n & 
       \jmath^i \vee 
       \overline{\ell'^i_1} \vee \ldots \vee \overline{\ell'^i_n} \vee \ell^i
      }
$$

\bigskip

%%% Conflict inference rule
\textbf{Conflict:}
$$
\infer[\confl{i}{\sigma}]
      {(\jmath_1^i \sigma) \vee (\jmath_2^i \sigma)}
      {(\jmath_1^i \vee \ell^i) & (\jmath_2^i \vee \overline{\ell'^i})}
$$ 

\bigskip

%%% CDCL inference rule
\textbf{Conflict-Driven Clause Learning:} 
$$
\infer[\cdcl{i}{p}]
      { \jmath \vee (\overline{\ell^i_1} \sigma^1_1 \vee \ldots \vee \overline{\ell^i_1} \sigma^1_{m_1}) \vee \ldots \vee (\overline{\ell^i_n} \sigma^n_1 \vee \ldots \vee \overline{\ell^i_n} \sigma^n_{m_n})
      }
	  { \infer*{\jmath}
               { \infer*[({\sigma'}_1^1,\ldots,{\sigma'}_{m'_1}^1)]
                      {}{\jmath^i_1} &  &
                 \infer*[({\sigma'}_1^1,\ldots,{\sigma'}_{m'_1}^1)]
                      {}{\jmath^i_m} &  &
                 \infer*[(\sigma_1^1,\ldots,\sigma_{m_1}^1)]
                      {}{[\ell^i_1]^{\dls{p}{1} } } &  &
                 \infer*[(\sigma_1^n,\ldots,\sigma_{m_n}^n)]
                      {}{[\ell^i_n]^{\dls{p}{n}}}
               }
      }
$$ 
где $\jmath = ({\jmath^i_1} \sigma'^1_1 \vee \ldots \vee \jmath^i_1 \sigma'^1_{m'_1}) \vee \ldots \vee ({\jmath^i_k} \sigma'^k_1 \vee \ldots \vee \jmath^i_k\sigma'^k_{m'_k})$

\bigskip

%%% Knowledge sharing inference rule
\textbf{Knowledge sharing:} 
$$
\infer[\kshare{i}{j}]
      {\phi^j}
	  {\phi^i}
$$

где $\phi^a$ --- клоз из актора $a$

\end{calculus}

\caption{Исчисление \emph{Expertised Conflict Resolution}}

\label{fig:ECR}
\end{figure}

\section{Доказательство полноты по опровержению}

Из \cite{DBLP:journals/corr/SlaneyP16} мы знаем, что \emph{Conflict Resolution} полно относительно \emph{Resolution}, а значит достаточно показать, что \emph{Expertised Conflict Resolution} симулирует \emph{Conflict Resolution}.

\subsection{Доказательство для части Бернайса-Шейфенкеля}

На данном этапе пример \ref{exmpl:cdcl-fail} не является проблемой, поскольку правило \emph{Knowledge sharing} позволяет нам обмениваться клозами между акторами, и никак не препятствует выводу \emph{CDCL}. Однако, для реализации это важный момент, который будет рассмотрен в третьей главе.

\begin{lemma}
\label{lem:upr}
Правило \emph{Unit-Propagation Resolution} из \emph{Conflict Resolution} можно выразить в исчислении \emph{Expertised Conflict Resolution}.
\end{lemma}
\begin{proof}
Обозначим за $L$ множество литералов $\{\ell_1, \ldots, \ell_n\}$ из правила \emph{Unit Propagation Resolution} на рис. \ref{fig:CR}. $\ell$ --- выводимый литерал.
Доказательство в \emph{ECR} будем строить индукцией по размеру множества $L$:
\begin{itemize}[label=$\star$]
	\item \emph{База:} Существует такой актор $i$, что для любого литерала $\ell_k$ из $L$ верно, что актор $i$ специализируется на этом литерале, т.е. $\Psi_i(\ell_k)$. Если $\Psi_i(\ell)$, то в акторе $i$ можно применить правило \emph{Unit-Propagation Resolution}, описанное на рис. \ref{fig:ECR}, это правило даст нужное нам $\ell$. В противном случае правило \emph{Conflict} сократит все литералы, кроме выводимого $\ell$.
    \item \emph{Переход:} Так как $L$ --- конечное непустое множество, то можно выбрать произвольный литерал $\ell'$ из $L$, и актор $j$, который специализируется на этом литерале. Бывает два случая:
    \begin{enumerate}
    	\item Актор $j$ специализируется на выводимом литерале $\ell'$. Тогда в этом акторе можно применить правило \emph{Unit-Propagation Resolution}, которое даст $\ell \vee \jmath$, где $\jmath$ --- остаточный клоз, а множество $L$, соответствующее $\jmath$, будет содержать строго меньшее количество литералов.
        \item Актор $j$ не специализируется на $\ell'$. Тогда в этом акторе можно применить правило \emph{Conflict}, которое даст результат, аналогичный пункту (a).
    \end{enumerate} 
\end{itemize}
\end{proof}

\begin{lemma}
\label{lem:confl}
Правило \emph{Conflict} из \emph{Conflict Resolution} можно выразить в исчислении \emph{Expertised Conflict Resolution}.
\end{lemma}
\begin{proof}
В правиле \emph{Conflict} на рис. \ref{fig:CR} участвуют литералы $\ell$ и $\overline{\ell'}$. По свойству разбиения специализации по акторам, существует такой актор $j$, что $\Psi_j(\ell)$ и $\Psi_j(\overline{\ell'})$, а значит в нём правило \emph{Conflict} из \emph{EPR}, по $\ell^j\text{ и }(\overline{\ell'})^j$ выведет $\bot$.
\end{proof}

\begin{lemma}
\label{lem:cdcl}
Правило \emph{Conflict-Driven Clause Learning} из \emph{Conflict Resolution} можно выразить в исчислении \emph{Expertised Conflict Resolution}.
\end{lemma}
\begin{proof}
Правило \emph{Conflict-Driven Clause Learning} из исчисления \emph{CR} было описано на рисунке \ref{fig:CR}. Пусть в нём были сделаны предположения $D = \{\ell_1, \ldots, \ell_n\}$. Индукцией по размеру множества предположений мы можем назначить каждому литералу идентификатор актора, который специализируется на этом литерале, и в котором будет сделано данное предположение в доказательстве из исчисления \emph{ECR}. В \emph{CR} из множества $D$ выводится $\bot$, а значит по лемме \ref{lem:confl} существует вывод $\bot$ и в \emph{ECR}. После применения \ref{lem:confl} мы можем применить правило \emph{Conflict-Driven Clause Learning} из исчисления \emph{ECR}, и таким образом получить $\ell_1 \vee \ldots \vee \ell_n$.
\end{proof}

\begin{theorem}
Для любой формулы $\phi$, выводимой в \emph{Conflict Resolution}, существует номер актора $i$ и вывод формулы $\phi^i$ в \emph{Expertised Conflict Resolution}.
\end{theorem}
\begin{proof}
Заметим, что благодаря правилу \emph{Knowledge sharing}, если вершина выводима в акторе $i$, то она выводима в любом акторе $j$. Поэтому мы можем доказывать более сильное утверждение:
пусть $\psi$ --- вывод клоза $\phi$ из множества клозов $S$ в исчислении \emph{Conflict Resolution}. Покажем, что для любого $i$ существует вывод клоза $\phi^i$ из множества клозов $S$ в новом исчислении \emph{Expertised Conflict Resoution} индукцией по структуре доказательства в \emph{Conflict Resolution}.

\begin{itemize}[label=$\star$] 
\item \emph{База:} Единственная вершина $c$, содержащая формулу $\phi$ --- по начальному распределению она попадёт в какой-то актор $i$, а значит есть вывод $\phi^i$, следовательно есть вывод $\phi^j$ для произвольного $j$.
\item \emph{Переход:} Вершина $c$ получена из вершин $\{a_1, \ldots, a_n\}$, в которых выведены формулы $\{\phi_1, \ldots, \phi_n\}$ с помощью одного из правил: \emph{Unit-Propagation Resolution}, \emph{Conflict}, \emph{Conflict-Driven Clause Learning}. Тогда по индуктивному переходу для произвольного актора $i$ существует доказательство формул $\{\phi_1^i, \ldots, \phi_n^i\}$. Для каждого из правил мы доказали лемму, которая строит вывод этого правила в \emph{ECR}, зная доказательство в \emph{CR}: 
\begin{enumerate}
	\item Для правила \emph{Unit-Propagation Resolution} воспользуемся леммой \ref{lem:upr}.
    \item Для правила \emph{Conflict} воспользуемся леммой \ref{lem:confl}.
    \item Для правила \emph{Conflict-Driven Clause Learning} воспользуемся леммой \ref{lem:cdcl}.
\end{enumerate}
\end{itemize}
\end{proof}

% \subsection{Доказательство для общего случая}

% \section{Динамическое изменение}
% \label{sec:onlineChanging}


\section{Выводы ко второй главе}

\begin{enumerate}
	\item Введено новое исчисление \emph{Expertised Conflict Resolution}, которое обобщает \emph{Conflict Resolution} на модель акторов.
    \item Доказана полнота по опровержению исчисления \emph{ECR}.
\end{enumerate}
\chapter{Экспериментальные данные}
\startrelatedwork

\textbf{TODO: показать графики, сравнить с Scavenger 0.1}


\printmainbibliography

\end{document}
